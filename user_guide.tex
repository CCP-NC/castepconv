\documentclass[10pt]{article}
\usepackage{listings}

\title{CASTEP convergence automation tool 1.0 User Guide}
\author{Simone Sturniolo}

\begin{document}
\maketitle

\section{Introduction}

The CASTEP convergence automation tool (CASTEPconv) is a Python script designed 
to automate the process of calculating the convergence of energy and forces in 
DFT calculations with CASTEP, as a function of the cutoff energy and the number 
of kpoints used. It works with Python 2.6 or higher. Given a .cell file, the 
script is able to create a set of folders containing input files for the 
required simulations, run the jobs, and process the output to produce tabulated 
ASCII data files and graphs. A .param file can be introduced as well to define 
any additional options for the DFT calculation, while a .conv file can be used 
to provide further options for the automated convergence calculation itself.

\section{Installation}

To install the script on Linux or CygWin just open a terminal, navigate to the 
folder containing the files, and type:

\begin{lstlisting}[language=Bash]
 user@machine:~$ sudo python setup.py install
\end{lstlisting}

Enter your password as prompted and the script will be installed and available 
as a command.

\section{Usage}

CASTEPconv requires only a .cell file to be present in the folder when it's 
used. Considering a file named ``\textless \textit{seedname}\textgreater.cell'', 
where \textless \textit{seedname}\textgreater represents the name of the job, 
the syntax to run the convergence job is simply

\begin{lstlisting}[language=Bash]
 user@machine:~$ castepconv.py [options] <seedname>
\end{lstlisting}

while in the folder where the file is kept. If one wants the convergence DFT 
simulations to have additional parameters (e.g. redefine the convergence 
criteria for SCF iterations, add dispersion correction etc.), a \textless 
\textit{seedname}\textgreater.param file can be added in the same folder with 
these options. To control the convergence job itself, instead, a new file, 
\textless \textit{seedname}\textgreater.conv, is needed.\newline
Options before the seedname can be used to quickly override settings in the .conv file. At the moment only one option is accepted, to override the \textbf{convergence\_task} keyword (see ``Options'' section).

\section{Fine grid convergence (added in version 0.9.5)}

Starting from version 0.9.5, besides cut-off energy and k-point grid, the possibility to converge the fine Gmax parameter has been added to CASTEPconv.\newline
Fine Gmax is a different way to express what is also known as 'fine grid' in CASTEP, namely a finer spatial grid than the one on which the electronic wavefunctions are calculated, used mainly to project the electronic density in the pseudopotential region. This grid size can be defined in CASTEP either as a multiple of the size of the standard grid (which, in turn, is controlled by the cut-off) by using the parameter \textit{fine\_grid\_scale} or with a completely independent parameter \textit{fine\_gmax} which is fundamentally equivalent to a cut-off radius, except for the fact that it is expressed in terms of a reciprocal wavelength rather than an energy. This parameter can be considered as the radius of a sphere in the reciprocal space, exactly like the cut-off; and since it describes a grid that is finer than the regular one in direct space, this sphere must be \textit{bigger} than the cut-off one in reciprocal space.\newline
To switch between the representation of this radius in terms of a cut-off energy and that as a reciprocal lattice vector, the regular DeBroglie wavelength formula applies:

\begin{equation}
 k_{cut} = \sqrt{\frac{2m_eE_{cut}}{\hbar^2}}
\end{equation}

where $E_{cut}$ is the cut-off energy and all terms are in SI units. CASTEP's default setting is that both the standard grid and the fine grid have in the reciprocal space a radius of $G_{max} = 1.75k_{cut}$, where $1.75$ is the default value of both the \textit{grid\_scale} and the \textit{fine\_grid\_scale} parameters. This is therefore the starting value and the lower boundary for fine Gmax at a given cutoff, since the fine grid can not have a smaller sphere than the standard one.\newline
CASTEPconv allows you to perform a convergence calculation on the fine Gmax parameter, starting from the minimum value at a given cutoff and increasing in steps of your choice. The parameters to do this are described extensively in the next section. It must be noted that, at a difference from what happens with k-point convergence, the chosen value of cut-off is NOT irrelevant to the result of this convergence: therefore you might want to perform a regular convergence first and only when you have a good guess of what the final cut-off to be used is proceed to perform a second convergence job on fine Gmax. It is possible to have some choice in the cut-off value used for the fine Gmax convergence as well. If you still have doubts about how to proceed, follow through the example in the Getting Started section to try it first hand.

\section{Convergence parameters}

The syntax of the .conv file is similar to the one of the CASTEP .param file:

\begin{lstlisting}
 <parameter name 1>:    <parameter value 1>
 <parameter name 2>:    <parameter value 2>
 ...
\end{lstlisting}

The accepted parameter names are listed in the subsections below.

\subsection{String parameters}

\textbf{convergence\_task}: describes the task that is required from the 
convergence script. This can be INPUT (creation of input files and folders), 
INPUTRUN (same as INPUT, plus actually runs the jobs), OUTPUT (processes some 
already finished calculations and creates output files), ALL (does all of the 
previous things, waiting for the jobs to finish before processing the output) and CLEAR (deletes all files from previous jobs).\newline
Default is INPUT.\newline
\textit{Legal values}: INPUT, INPUTRUN, OUTPUT, ALL, CLEAR\newline

\textbf{running\_mode}: describes the mode in which the calculations should be 
ran. If PARALLEL, all calculations will be launched at the same time (ideal for 
job submission on a cluster). If SERIAL, the program will wait for one 
calculation to finish before the next one begins (and reuse the .check file from 
the previous calculation as a starting point). In the latter case, all files 
will be created in a single folder. Default is SERIAL.\newline
\textit{Legal values}: PARALLEL, SERIAL\newline

\textbf{output\_type}: plotting program for which an output script should be 
created. Both GNUPLOT and GRACE are supported. Default is GNUPLOT.\newline
\textit{Legal values}: GNUPLOT, GRACE\newline

\textbf{running\_command}: command line that should be used to run jobs. This 
should be expressed by replacing the name of the job with the generic token 
\textless \textit{seedname}\textgreater. Default is \textit{castep \textless 
seedname\textgreater -dryrun}.\newline
\textit{Legal values}: any string. If it contains the token \textless 
\textit{seedname}\textgreater, it will be appropriately replaced.\newline

\textbf{submission\_script}: only in version 0.9.2 and higher. The filename of a 
script that needs to be copied inside all folders before execution. This has 
been included to accomodate the issues with some large supercomputers that might 
require job submission to take place by means of such a script rather than a 
single command. In this case, the script will be renamed as the job, without any 
extensions, and any instances of \textless \textit{seedname}\textgreater inside 
it will be replaced by the job name as well. So for example by adding such a 
script (with the proper job configuration details) it would be possible to 
submit one's job on a system that supports \textit{qsub} by using \texttt{qsub 
\textless~\textless seedname\textgreater} as a 
\textit{running\_command}.\newline
\textit{Legal values}: any file name without spaces in it.

\textbf{fine\_gmax\_mode}: only in version 0.9.5 and higher. This parameter controls the value of the cut-off used when performing
fine Gmax convergence. If set to NONE, no fine Gmax convergence will be performed. If set to MIN, fine Gmax will be performed
with cutoff\_min as the cut-off. If set to MAX, cutoff\_max will be used instead. Default is NONE.
\newline
\textit{Legal values}: NONE, MIN, MAX

\subsection{Float parameters}

\textbf{cutoff\_min}: Minimum value for the cutoff range explored in eV. Default 
is 400.0 eV.\newline
\textit{Legal values}: Any positive float.\newline

\textbf{cutoff\_max}: Maximum value for the cutoff range explored in eV. Default 
is 800.0 eV.\newline
\textit{Legal values}: Any positive float greater than cutoff\_min.\newline

\textbf{cutoff\_step}: Step between the values of the cutoff range explored in 
eV. Default is 100.0 eV.\newline
\textit{Legal values}: Any positive float.\newline

\textbf{displace\_atoms}: Displacement in Angstroms to introduce in atom 
positions - necessary when the cell is equilibrated and it is not possible to 
converge forces because they are zero. Default is 0.0 Ang.\newline
\textit{Legal values}: Any float\newline

\textbf{final\_energy\_delta}: Tolerance on final energy for the estimate of 
convergence. Default is 0.00001 eV/atom.\newline
\textit{Legal values}: Any positive float\newline

\textbf{forces\_delta}: Tolerance on maximum force for the estimate of 
convergence. See above. Default is 0.05 eV/Ang.\newline
\textit{Legal values}: Any positive float\newline

\textbf{stresses\_delta}: Tolerance on maximum stress for the estimate of 
convergence. See above. Default is 0.1 GPa.\newline
\textit{Legal values}: Any positive float\newline

\textbf{fine\_gmax\_min}: Minimum value for the fine Gmax range explored in eV. Default 
is the minimum value for the used cutoff ($3.0625E_{cut}$). It is ignored if fine\_gmax\_mode is NONE.\newline
\textit{Legal values}: Any positive float greater or equal to $3.0625E_{cut}$.\newline

\textbf{fine\_gmax\_max}: Maximum value for the fine Gmax range explored in eV. Default 
is fine\_gmax\_min + 3fine\_gmax\_step. It is ignored if fine\_gmax\_mode is NONE.\newline
\textit{Legal values}: Any positive float greater than $3.0625E_{cut}$.\newline

\textbf{fine\_gmax\_step}: Step between the values of fine Gmax explored in 
eV. Default is 100.0 eV. It is ignored if fine\_gmax\_mode is NONE.\newline
\textit{Legal values}: Any positive float.\newline

\subsection{Integer parameters}

\textbf{kpoint\_n\_min}: Minimum value for the k-point range explored. This 
applies to the shortest side of the kpoint\_mp\_grid: depending on the size of 
the other cell parameters, there might be proportionally more k-points along 
other sides. Default is 1.\newline
\textit{Legal values}: Any positive integer.\newline

\textbf{kpoint\_n\_max}: Maximum value for the k-point range explored. Default 
is 4.\newline
\textit{Legal values}: Any positive integer greater than kpoint\_n\_min.\newline

\textbf{kpoint\_n\_step}: Step between the values of the k-point range explored. 
Default is 1.\newline
\textit{Legal values}: Any positive integer.\newline

\textbf{max\_parallel\_jobs}: Maximum number of parallel jobs to run when in 
``parallel'' mode. Ignored in ``serial'' mode. Zero means that there is no 
limit. Default is 0.\newline
\textit{Legal values}: Any non negative integer (negative values will be 
ignored).\newline

\subsection{Boolean parameters}

\textbf{converge\_stress}: Apply calculation of stresses to the simulations and 
then estimate convergence on stresses as well as energy and forces. Default is 
False.\newline
\textit{Legal values}: Anything. The word ``true'', regardless of the case, 
means the stresses are calculated. Anything else will be interpreted as 
False.\newline

\textbf{reuse\_calcs}: If results from a previous convergence are present, 
recycle them when possible. This requires the .conv\_tab file from the previous 
calculation to be unaltered. Default is False.\newline
\textit{Legal values}: Anything. The word ``true'', regardless of the case, 
means the stresses are calculated. Anything else will be interpreted as 
False.\newline

\textbf{serial\_reuse}: When running a serial calculation, reuse the .check file from previous calculations to start your new simulations instead of starting from scratch to speed up SCF convergence. Default is True.\newline
\textit{Legal values}: Anything. The word ``true'', regardless of the case, 
means the serial reuse is employed. Anything else will be interpreted as 
False.\newline

\section{Options}

At the moment, CASTEPconv allows only for one command-line option. The option \textit{-t} allows one to quickly override the \textbf{convergence\_task} keyword in the conv file with another task of choice, using one of the following values as argument: \textit{i, ir, o, c, a} standing respectively for Input, InputRun, Output, Clean and All. For example, to quickly clear the previously written files before restarting a calculation one can quickly issue the command:

\begin{lstlisting}
  user@machine:~$ castepconv.py -t c <seedname>
\end{lstlisting}

\section{Output}

When the calculations are over, CASTEPconv will have produced the following 
files:

\textbf{\textless \textit{seedname}\textgreater.conv\_tab}: this file is created 
during the INPUT phase of the run (when folders and input files for the 
calculations are created) and sums up the values used for cutoff and the kpoint 
grids employed in the various files. This is just useful as a memo, and mostly 
used to resume analysis of previously ran calculations (by using the OUTPUT 
convergence\_task) - if it's not present, the ranges will be recalculated from 
the .conv file.

\textbf{\textless \textit{seedname}\textgreater\_cut\_conv.dat}: generated in 
the OUTPUT phase of the run, will contain a tabulated ASCII of cutoff values (in 
eV), final energies (in eV) and maximum forces (in eV/Ang) for the various 
calculations ran. The maximum force is the highest modulus $|\mathbf{f}_i|$ 
between the forces acting on all atoms $i$ in the system. If 
\textbf{converge\_stress} has been set to True, a fourth column will contain the 
value of the maximum stress component acting on the system.

where $\sigma_{ij}$ is the generic element of the stress tensor.

\textbf{\textless \textit{seedname}\textgreater\_kpn\_conv.dat}: generated in 
the OUTPUT phase of the run, it is similar to the above, except that it 
expresses its quantities as a function of the total number of kpoints in the 
grid.

\textbf{\textless \textit{seedname}\textgreater\_conv.\textless variable 
extension\textgreater}: this file is the script meant for generation of graphic 
output. By default it will be a Gnuplot script (extension .gp). When other forms 
of output will be supported, choosing the appropriate value for the output\_type 
option will replace it with a different format.

\textbf{\textless \textit{seedname}\textgreater\_report.txt}: only in version 0.9.5 and higher. This file contains a printout of the same suggestions for the optimal convergence values that will be printed on the standard output, for future reference.

Besides this, CASTEPconv produces a bit of textual output to suggest which 
minimum values of cutoff and k-point grid might be the best. This is done by 
taking the difference between final energy, maximum force, and if required 
maximum stress, for successive values and comparing it with a tolerance which, 
by default, is equal to the tolerance assumed in CASTEP calculations on the same 
quantities. It should be kept to mind that this is only a rule of thumb method, 
and that it needs to be considered only as an indication - by no means this is 
meant to be always the correct answer. Different simulations will also require 
different convergence criteria. Whenever no convergence is found, or the values 
examined are too small to be sure of their reliability, a warning message is 
issued.

\section{Getting Started}

In this section we'll go through a quick tutorial to getting started with CASTEPconv, by using the silicon example provided with the code. Even when the example includes running calculations, they should be fairly simple even for a desktop computer. If you are using an especially old or slow machine, please skip those parts.\newline
In order to run the example, just enter the console, move to the \textit{castepconv/example} folder and edit the \textit{Si.conv} file with a text editor. Replace the line

\begin{lstlisting}
 running_command: castep.serial <seedname>
\end{lstlisting}

with whatever the name/path of castep on your system is (remember to keep the \textless \textit{seedname}\textgreater part untouched though). After doing that, you can just use the command

\begin{lstlisting}
 user@machine:~$ castepconv.py Si
\end{lstlisting}

to run the convergence. Just sit back and wait. At the end of the process, after a few minutes, some output files should have been produced. The ones of interest are \textit{Si\_cut\_conv.dat, Si\_cut\_conv.gp, Si\_kpn\_conv.dat} and \textit{Si\_kpn\_conv.gp}. The two .dat files contain tabulated results for energy and forces by varying, respectively, cutoff and k-point grid; the two .gp files are scripts for Gnuplot plotting of the same data sets. If you have gnuplot installed they can be visualized for example with:

\begin{lstlisting}
 user@machine:~$ gnuplot Si_cut_conv.gp
\end{lstlisting}

If you plot the data, you will notice that the line for the ``force'' is completely flat. This happens because the \textit{Si.cell} file used describes a perfectly equilibrated structure, and in a system in equilibrium, forces are always zero. In order to change this we need to upset this equilibrium a bit. To do so go back to editing the \textit{Si.conv} file and uncomment (remove the hash, \#, from the beginning) the line

\begin{lstlisting}
displace_atoms: 0.05
\end{lstlisting}

which will shift all atoms by $0.05 $\AA{} in a random direction, enough to give rise to non zero forces. Now rerun the calculation, and when prompted, answer to overwrite the old files. At the end you will be able to see the convergence of forces as well.\newline
If instead you'd like to run everything by hand and let CASTEPconv handle only the input generation and the output postprocessing, that is possible too. First let's clean up the current folder structure by issuing the command

\begin{lstlisting}
 user@machine:~$ castepconv.py -t c Si
\end{lstlisting}

which replaces the current task (\textit{all}) with \textit{clean}, namely, delete all generated files and folders. Then edit the \textit{Si.conv} file and uncomment the following lines:

\begin{lstlisting}
running_mode: parallel
\end{lstlisting}

and either replace \textit{convergence\_task} with \textit{input} and run normally, or run with the option \textit{-t i}. In this way, a structure of folders will be generated, containing the starting files for the various points of the convergence calculations. This could be done with the \textit{serial} running mode as well, but in that case everything would be placed in a single folder and you would need to run the tasks in a specific order for the calculations to succeed (as every calculation in that case reuses the results from the previous one as a starting point in order to improve its performance). After generating the folders, visit each one of them and run the CASTEP calculation normally. When all calculations are finished, go back in the parent folder and either change \textit{convergence\_task} with \textit{output} and run normally or run with the option \textit{-t o} in order to get the output postprocessed - as an outcome, you will get the same data files and plots as in the first run.

\end{document}

