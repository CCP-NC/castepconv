\documentclass[10pt]{article}
\usepackage{listings}

\title{CASTEP convergence automation tool 1.0 User Guide}
\author{Simone Sturniolo}

\begin{document}
\maketitle

\section{Introduction}

The CASTEP convergence automation tool (CASTEPconv) is a Python script designed to automate the process of calculating the convergence of energy and forces in DFT calculations with CASTEP, as a function of the cutoff energy and the number of kpoints used. It works with Python 2.6 or higher. Given a .cell file, the script is able to create a set of folders containing input files for the required simulations, run the jobs, and process the output to produce tabulated ASCII data files and graphs. A .param file can be introduced as well to define any additional options for the DFT calculation, while a .conv file can be used to provide further options for the automated convergence calculation itself.

\section{Installation}

To install the script on Linux or CygWin just open a terminal, navigate to the folder containing the files, and type:

\begin{lstlisting}[language=Bash]
 user@machine:~$ sudo python setup.py install
\end{lstlisting}

Enter your password as prompted and the script will be installed and available as a command.

\section{Usage}

CASTEPconv requires only a .cell file to be present in the folder when it's used. Considering a file named ``\textless \textit{seedname}\textgreater.cell'', where \textless \textit{seedname}\textgreater represents the name of the job, the syntax to run the convergence job is simply

\begin{lstlisting}[language=Bash]
 user@machine:~$ castepconv.py [--task={c, i, ir, o, a}] <seedname>
\end{lstlisting}

while in the folder where the file is kept. The command line option \textit{--task} allows one to determine which kind of task the script has to perform (refer to the subsection on string parameters for more details); this setting overrides the one contained in the configuration file. If one wants the convergence DFT simulations to have additional parameters (e.g. redefine the convergence criteria for SCF iterations, add dispersion correction etc.), a \textless \textit{seedname}\textgreater.param file can be added in the same folder with these options. To control the convergence job itself, instead, a new file, \textless \textit{seedname}\textgreater.conv, is needed.\newline
When running a new convergence job with existing folders from a previous job, ome must remember that \textit{all .castep and .check files will be deleted}, unless the parameter \textbf{reuse\_calcs} has been set to True. If for some reason one wants to keep the old files, they need to be renamed or moved before the new job is launched.

\section{Convergence parameters}

The syntax of the .conv file is similar to the one of the CASTEP .param file:

\begin{lstlisting}
 <parameter name 1>:    <parameter value 1>
 <parameter name 2>:    <parameter value 2>
 ...
\end{lstlisting}

The accepted parameter names are listed in the subsections below.

\subsection{String parameters}

\textbf{convergence\_task}: describes the task that is required from the convergence script. This can be INPUT (creation of input files and folders), INPUTRUN (same as INPUT, plus actually runs the jobs), OUTPUT (processes some already finished calculations and creates output files), CLEAR (deletes all existing folders and files generated by previous runs) and ALL (does all of the previous things, waiting for the jobs to finish before processing the output). Default is INPUT.\newline
\textit{Legal values}: CLEAR, INPUT, INPUTRUN, OUTPUT, ALL\newline

\textbf{running\_mode}: describes the mode in which the calculations should be ran. If PARALLEL, all calculations will be launched at the same time (ideal for job submission on a cluster). If SERIAL, the program will wait for one calculation to finish before the next one begins (and reuse the .check file from the previous calculation as a starting point). In the latter case, all files will be created in a single folder. Default is PARALLEL.\newline
\textit{Legal values}: PARALLEL, SERIAL\newline

\textbf{output\_type}: plotting program for which an output script should be created. For now only GNUPLOT and GRACE are supported. The keywords GRACE and XMGRACE are equivalent. Default is GNUPLOT.\newline
\textit{Legal values}: GNUPLOT, [XM]GRACE\newline

\textbf{running\_command}: command line that should be used to run jobs. This should be expressed by replacing the name of the job with the generic token \textless \textit{seedname}\textgreater. Default is \textit{castep \textless seedname\textgreater -dryrun}.\newline
\textit{Legal values}: any string containing the token \textless \textit{seedname}\textgreater\newline

\subsection{Float parameters}

\textbf{cutoff\_min}: Minimum value for the cutoff range explored in eV. Default is 400.0 eV.\newline
\textit{Legal values}: Any positive float.\newline

\textbf{cutoff\_max}: Maximum value for the cutoff range explored in eV. Default is 800.0 eV.\newline
\textit{Legal values}: Any positive float greater than cutoff\_min.\newline

\textbf{cutoff\_step}: Step between the values of the cutoff range explored in eV. Default is 100.0 eV.\newline
\textit{Legal values}: Any positive float.\newline

\textbf{displace\_atoms}: Displacement in Angstroms to introduce in atom positions - necessary when the cell is equilibrated and it is not possible to converge forces because they are zero. Default is 0.0 Ang.\newline
\textit{Legal values}: Any float\newline

\textbf{final\_energy\_delta}: Tolerance on final energy for the estimate of convergence. Default is 0.00001 eV/atom.\newline
\textit{Legal values}: Any positive float\newline

\textbf{forces\_delta}: Tolerance on maximum force for the estimate of convergence. See above. Default is 0.05 eV/Ang.\newline
\textit{Legal values}: Any positive float\newline

\textbf{stresses\_delta}: Tolerance on maximum stress for the estimate of convergence. See above. Default is 0.1 GPa.\newline
\textit{Legal values}: Any positive float\newline

\subsection{Integer parameters}

\textbf{kpoint\_n\_min}: Minimum value for the k-point range explored. This applies to the shortest side of the kpoint\_mp\_grid: depending on the size of the other cell parameters, there might be proportionally more k-points along other sides. Default is 1.\newline
\textit{Legal values}: Any positive integer.\newline

\textbf{kpoint\_n\_max}: Maximum value for the k-point range explored. Default is 4.\newline
\textit{Legal values}: Any positive integer greater than kpoint\_n\_min.\newline

\textbf{kpoint\_n\_step}: Step between the values of the k-point range explored. Default is 1.\newline
\textit{Legal values}: Any positive integer.\newline

\textbf{max\_parallel\_jobs}: Maximum number of parallel jobs to run when in ``parallel'' mode. Ignored in ``serial'' mode. Zero means that there is no limit. Default is 0.\newline
\textit{Legal values}: Any non negative integer (negative values will be ignored).\newline

\subsection{Boolean parameters}

\textbf{converge\_stress}: Apply calculation of stresses to the simulations and then estimate convergence on stresses as well as energy and forces. Default is False.\newline
\textit{Legal values}: Anything. The word ``true'', regardless of the case, means the stresses are calculated. Anything else will be interpreted as False.\newline

\textbf{reuse\_calcs}: If results from a previous convergence are present, recycle them when possible. This requires the .conv\_tab file from the previous calculation to be unaltered. Default is False.\newline
\textit{Legal values}: Anything. The word ``true'', regardless of the case, means the stresses are calculated. Anything else will be interpreted as False.\newline

\textbf{serial\_reuse}: When a serial convergence job is ran, include the keyword ``reuse'' in the .param files in order to make following calculations reuse the data from .check files from the previous ones. Turning this option off will make the jobs slower but may avoid crashes on some older versions of CASTEP. Default is True.\newline
\textit{Legal values}: Anything. The word ``true'', regardless of the case, means the stresses are calculated. Anything else will be interpreted as False.\newline

\section{Output}

When the calculations are over, CASTEPconv will have produced the following files:

\textbf{\textless \textit{seedname}\textgreater.conv\_tab}: this file is created during the INPUT phase of the run (when folders and input files for the calculations are created) and sums up the values used for cutoff and the kpoint grids employed in the various files. This is just useful as a memo, and mostly used to resume analysis of previously ran calculations (by using the OUTPUT convergence\_task) - if it's not present, the ranges will be recalculated from the .conv file.

\textbf{\textless \textit{seedname}\textgreater\_cut\_conv.dat}: generated in the OUTPUT phase of the run, will contain a tabulated ASCII of cutoff values (in eV), final energies (in eV) and maximum forces (in eV/Ang) for the various calculations ran. The maximum force is the highest modulus $|\mathbf{f}_i|$ between the forces acting on all atoms $i$ in the system. If \textbf{converge\_stress} has been set to True, a fourth column will contain the value of the maximum stress component acting on the system.

\textbf{\textless \textit{seedname}\textgreater\_kpn\_conv.dat}: generated in the OUTPUT phase of the run, it is similar to the above, except that it expresses its quantities as a function of the total number of kpoints in the grid.

\textbf{\textless \textit{seedname}\textgreater\_conv.\textless variable extension\textgreater}: this file is the script meant for generation of graphic output. By default it will be a Gnuplot script (extension .gp). When other forms of output will be supported, choosing the appropriate value for the output\_type option will replace it with a different format.

Besides this, CASTEPconv produces a bit of textual output to suggest which minimum values of cutoff and k-point grid might be the best. This is done by taking the difference between final energy, maximum force, and if required maximum stress, for successive values and comparing it with a tolerance which, by default, is equal to the tolerance assumed in CASTEP calculations on the same quantities. It should be kept to mind that this is only a rule of thumb method, and that it needs to be considered only as an indication - by no means this is meant to be always the correct answer. Different simulations will also require different convergence criteria. Whenever no convergence is found, or the values examined are too small to be sure of their reliability, a warning message is issued.

\end{document}

